%%%%%%%%%%%%%%%%%%%%%%%%%%%%%%%%%%%%%%%%%
% CSE 2100 Laboratory Report
% LaTeX Template
%
% This template is a modified version of the original downloaded from:
% http://www.LaTeXTemplates.com
%
% Original author:
% Linux and Unix Users Group at Virginia Tech Wiki 
% (https://vtluug.org/wiki/Example_LaTeX_chem_lab_report)
%
% Modified by cmcmurrough & Gergely Zaruba
%
% License:
% CC BY-NC-SA 3.0 (http://creativecommons.org/licenses/by-nc-sa/3.0/)
%
%%%%%%%%%%%%%%%%%%%%%%%%%%%%%%%%%%%%%%%%%

%----------------------------------------------------------------------------------------
%	PACKAGES AND DOCUMENT CONFIGURATIONS
%----------------------------------------------------------------------------------------

\documentclass{article}

\usepackage{graphicx} % Required for the inclusion of images
\usepackage{natbib} % Required to change bibliography style to APA
\usepackage{amsmath} % Required for some math elements 

\setlength\parindent{0pt} % Removes all indentation from paragraphs

\renewcommand{\labelenumi}{\alph{enumi}.} % Make numbering in the enumerate environment by letter rather than number (e.g. section 6)

%\usepackage{times} % Uncomment to use the Times New Roman font

%----------------------------------------------------------------------------------------
%	DOCUMENT INFORMATION
%----------------------------------------------------------------------------------------

\title{Lab 8: CMake Revisited and GUIs \\ CSE 2100-001} % Title and course section

\author{Thomas Vu & John Jones} % Author name

\date{\today} % Report submission date

\begin{document}

\maketitle % Insert the title, author and date

\begin{center}
\begin{tabular}{l r}
Date Performed: & March 26, 2020 \\ % Date of the original lab session
Partners: & Thomas Vu \\ % Partner names
& John Jones \\
\end{tabular}
\end{center}

% If you wish to include an abstract, uncomment the lines below
% \begin{abstract}
% Abstract text
% \end{abstract}

%----------------------------------------------------------------------------------------
%	SECTION 1
%----------------------------------------------------------------------------------------

\section{Objective}

Watch both videos posted on YouTube for Lab-8. The first video shows how you should organize your code when working on small to medium sized projects with cmake. The second video swalks through building a dummy user interface.\\

Look at the simple GUI and corresponding code in the GUI\_Calculator/ Simple\_Calculator folder.
Familiarize yourself with the directory structure. Look at the CMakeLists.txt file. Study the main.cpp and global.h codes. Build the simple calculator from the build subdirectory (cmake .. ; make).\\

Look at the more useful calculator in GUI\_Calculator/ Calculator. This is a sample of what we expect you to create. We are providing the skeleton for both the GUI builder and the actual GUI code in GUI\_Calculator/ Skeleton\_Calculator. Use the build directory for building only, any edits should be done in the \textit{src} and \textit{include} directories. After changing the glade file in the src directory, the\\ \textit{cmake ..} \\command in the build directory will copy the glade file over into the build directory. A clean build can be obtained by issuing the command:\\ \textit{rm -rf *} \\from within the build directory. Be careful with this command as it erases everything from inside the directory in which it was issued! \\

In addition to turning in a completed version of this document on blackboard, you will need to turn in a .tgz archive of the completed calculator from the Skeleton\_Calculator folder. Once you have a working calculator with which you are happy, make sure that your build directory is empty (from inside the build directory, issue: \textit{rm -rf * }). Then from the main Skeleton\_Calculator directory issue the\\ \textit{tar -cvzf - * \textgreater\textasciitilde/My\_Calculator.tgz} \\
command. This will create the archive My\_Calculator.tgz in your home directory. You will need to upload this to blackboard.

\subsection{Definitions and Quick Questions}
\label{definitions}
\begin{description}
\item[GUI Builder:]
Software that takes the complexity out of manually building a GUI from code. This software allows the user to visibly divide the window area and arrange widgets accordingly.   
\item[pkg-config:]
Software that determines what libraries are needed for a particular program to run and uses this information to create a proper makefile.
\item[The two parameters of pkg\_check\_modules in CMakeLists.txt:]
PREFIX and Flags.
\item[tree:]
.
├── build
├── CMakeLists.txt
├── include
│   └── global.h
└── src
    ├── calculator.glade
    ├── global.cpp
    └── main.cpp

\item[]
\end{description} 

%----------------------------------------------------------------------------------------
%	SECTION 2
%----------------------------------------------------------------------------------------
\section{Question-set 1 -- CMake}
\textbf{Let us consider adding an object to our project (we use cmake to create Makefiles). We have defined the object in my\_object.h and provide the implementation details (e.g., constructors, destructors, member functions) in my\_object.cpp. Where in the directory structure would you place these two files? }\\

my_object.h goes into the include folder and my_object.cpp goes into the src folder. \\


\textbf{Let us further assume that in the implementation of your new object you are using functions from a third party library called libmatrix. You know that the header you are including for this (matrix.h) is located in /usr/include/libmatrix/. Since this is a library the linker should also link libmatrix.a which is located in /usr/lib/libmatrix/. How would you need to modify the CMakeLists.txt file (include folders, library folders, linkables)?}\\

include directories(/usr/include/libmatrix/) and link directories(/usr/lib/libmatrix/)\\

\textbf{Digging more into libmatrix, you find that it came with a pkg-config description. In this case, how would you need to modify the CMakeLists.txt file (include folders, library folders, linkables)?}\\

target link libraries() and target include directories(). \\

%----------------------------------------------------------------------------------------
%	SECTION 3
%----------------------------------------------------------------------------------------
\section{Question-set 2 -- GUIs}
\textbf{You use glade to make a user interface and write c++ code that loads it, displays it, and uses it. It works perfectly on your computer. You give the executable to your buddy who is running the same OS that you do. He complains that your code throws an error when starting up. What likely happened? }\\

Glade and/or Gtk+ might not be installed correctly. \\


\textbf{Should any of your event handling callback functions contain infinite loops or sleep (or sleep-like) statements? Why? }\\

Event handling should have sleep statements that wait for button or key presses. \\

\textbf{You made a GTK+3 based user interface (i.e., in your main code you turn over execution to GTK by calling gtk\_main()). Do some research and describe what the use of the gdk\_threads\_timeout() function could be (hint: timed events). Can you think of (and describe) a specific scenario where that function (and events created by it) could be useful?}\\

One example of how time events could be useful is the development of a timed quiz application where events only have a certain amount of time before the next event is triggered. \\


\end{document}